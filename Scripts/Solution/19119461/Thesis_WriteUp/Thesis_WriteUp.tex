\documentclass[11pt,preprint, authoryear]{elsarticle}

\usepackage{lmodern}
%%%% My spacing
\usepackage{setspace}
\setstretch{1.2}
\DeclareMathSizes{12}{14}{10}{10}

% Wrap around which gives all figures included the [H] command, or places it "here". This can be tedious to code in Rmarkdown.
\usepackage{float}
\let\origfigure\figure
\let\endorigfigure\endfigure
\renewenvironment{figure}[1][2] {
    \expandafter\origfigure\expandafter[H]
} {
    \endorigfigure
}

\let\origtable\table
\let\endorigtable\endtable
\renewenvironment{table}[1][2] {
    \expandafter\origtable\expandafter[H]
} {
    \endorigtable
}


\usepackage{ifxetex,ifluatex}
\usepackage{fixltx2e} % provides \textsubscript
\ifnum 0\ifxetex 1\fi\ifluatex 1\fi=0 % if pdftex
  \usepackage[T1]{fontenc}
  \usepackage[utf8]{inputenc}
\else % if luatex or xelatex
  \ifxetex
    \usepackage{mathspec}
    \usepackage{xltxtra,xunicode}
  \else
    \usepackage{fontspec}
  \fi
  \defaultfontfeatures{Mapping=tex-text,Scale=MatchLowercase}
  \newcommand{\euro}{€}
\fi

\usepackage{amssymb, amsmath, amsthm, amsfonts}

\def\bibsection{\section*{References}} %%% Make "References" appear before bibliography


\usepackage[round]{natbib}

\usepackage{longtable}
\usepackage[margin=2.3cm,bottom=2cm,top=2.5cm, includefoot]{geometry}
\usepackage{fancyhdr}
\usepackage[bottom, hang, flushmargin]{footmisc}
\usepackage{graphicx}
\numberwithin{equation}{section}
\numberwithin{figure}{section}
\numberwithin{table}{section}
\setlength{\parindent}{0cm}
\setlength{\parskip}{1.3ex plus 0.5ex minus 0.3ex}
\usepackage{textcomp}
\renewcommand{\headrulewidth}{0.2pt}
\renewcommand{\footrulewidth}{0.3pt}

\usepackage{array}
\newcolumntype{x}[1]{>{\centering\arraybackslash\hspace{0pt}}p{#1}}

%%%%  Remove the "preprint submitted to" part. Don't worry about this either, it just looks better without it:
\makeatletter
\def\ps@pprintTitle{%
  \let\@oddhead\@empty
  \let\@evenhead\@empty
  \let\@oddfoot\@empty
  \let\@evenfoot\@oddfoot
}
\makeatother

 \def\tightlist{} % This allows for subbullets!

\usepackage{hyperref}
\hypersetup{breaklinks=true,
            bookmarks=true,
            colorlinks=true,
            citecolor=blue,
            urlcolor=blue,
            linkcolor=blue,
            pdfborder={0 0 0}}


% The following packages allow huxtable to work:
\usepackage{siunitx}
\usepackage{multirow}
\usepackage{hhline}
\usepackage{calc}
\usepackage{tabularx}
\usepackage{booktabs}
\usepackage{caption}


\newenvironment{columns}[1][]{}{}

\newenvironment{column}[1]{\begin{minipage}{#1}\ignorespaces}{%
\end{minipage}
\ifhmode\unskip\fi
\aftergroup\useignorespacesandallpars}

\def\useignorespacesandallpars#1\ignorespaces\fi{%
#1\fi\ignorespacesandallpars}

\makeatletter
\def\ignorespacesandallpars{%
  \@ifnextchar\par
    {\expandafter\ignorespacesandallpars\@gobble}%
    {}%
}
\makeatother

\newlength{\cslhangindent}
\setlength{\cslhangindent}{1.5em}
\newenvironment{CSLReferences}%
  {\setlength{\parindent}{0pt}%
  \everypar{\setlength{\hangindent}{\cslhangindent}}\ignorespaces}%
  {\par}


\urlstyle{same}  % don't use monospace font for urls
\setlength{\parindent}{0pt}
\setlength{\parskip}{6pt plus 2pt minus 1pt}
\setlength{\emergencystretch}{3em}  % prevent overfull lines
\setcounter{secnumdepth}{5}

%%% Use protect on footnotes to avoid problems with footnotes in titles
\let\rmarkdownfootnote\footnote%
\def\footnote{\protect\rmarkdownfootnote}
\IfFileExists{upquote.sty}{\usepackage{upquote}}{}

%%% Include extra packages specified by user
\usepackage{array}
\usepackage{caption}
\usepackage{graphicx}
\usepackage{siunitx}
\usepackage[normalem]{ulem}
\usepackage{colortbl}
\usepackage{multirow}
\usepackage{hhline}
\usepackage{calc}
\usepackage{tabularx}
\usepackage{threeparttable}
\usepackage{wrapfig}
\usepackage{adjustbox}
\usepackage{hyperref}

%%% Hard setting column skips for reports - this ensures greater consistency and control over the length settings in the document.
%% page layout
%% paragraphs
\setlength{\baselineskip}{12pt plus 0pt minus 0pt}
\setlength{\parskip}{12pt plus 0pt minus 0pt}
\setlength{\parindent}{0pt plus 0pt minus 0pt}
%% floats
\setlength{\floatsep}{12pt plus 0 pt minus 0pt}
\setlength{\textfloatsep}{20pt plus 0pt minus 0pt}
\setlength{\intextsep}{14pt plus 0pt minus 0pt}
\setlength{\dbltextfloatsep}{20pt plus 0pt minus 0pt}
\setlength{\dblfloatsep}{14pt plus 0pt minus 0pt}
%% maths
\setlength{\abovedisplayskip}{12pt plus 0pt minus 0pt}
\setlength{\belowdisplayskip}{12pt plus 0pt minus 0pt}
%% lists
\setlength{\topsep}{10pt plus 0pt minus 0pt}
\setlength{\partopsep}{3pt plus 0pt minus 0pt}
\setlength{\itemsep}{5pt plus 0pt minus 0pt}
\setlength{\labelsep}{8mm plus 0mm minus 0mm}
\setlength{\parsep}{\the\parskip}
\setlength{\listparindent}{\the\parindent}
%% verbatim
\setlength{\fboxsep}{5pt plus 0pt minus 0pt}



\begin{document}



%titlepage
\thispagestyle{empty}
\begin{center}
\begin{minipage}{0.75\linewidth}
    \centering
%Entry1
    {\uppercase{\huge Correlating Factors of U.S. Presidential Speeches
with Stock Market Movements: a Machine Learning Approach.\par}}
    \vspace{2cm}
%Author's name
    {\LARGE \textbf{Pablo Rees}\par}
    \vspace{1cm}
%University logo
\begin{center}
    \includegraphics[width=0.3\linewidth]{Stellenbosch\_University\_Logo.jpeg}
\end{center}
\vspace{1cm}
%Supervisor's Details
\begin{center}
    {\Large Second draft formally presented with intention of fulfilment
of the requirements for completion of the MCom Economics Programme (full
thesis only) at Stellenbosch University.\par}
    \vspace{1cm}
%Degree
    {\large Under the supervision of: \vfill Dr Dawie van Lill\par}
    \vspace{1cm}
%Institution
    {\large Stellenbosch University Department of Economics\par}
    \vspace{1cm}
%Date
    {\large July 2022}
%More
    {\normalsize }
%More
    {\normalsize }
\end{center}
\end{minipage}
\end{center}
\clearpage


\begin{frontmatter}  %

\title{}

% Set to FALSE if wanting to remove title (for submission)

\begin{abstract}
\small{
Abstract to be written here. The abstract should not be too long and
should provide the reader with a good understanding what you are writing
about. Academic papers are not like novels where you keep the reader in
suspense. To be effective in getting others to read your paper, be as
open and concise about your findings here as possible. Ideally, upon
reading your abstract, the reader should feel he / she must read your
paper in entirety.
}
\end{abstract}

\vspace{1cm}


\begin{keyword}
\footnotesize{
S\&P 500 \sep Machine Learning \sep Natural Language Processing
\sep Presidential Speeches \sep Time Series Econometrics \\
\vspace{0.3cm}
}
\footnotesize{
\textit{JEL classification} G17
}
\end{keyword}



\vspace{0.5cm}

\end{frontmatter}



%________________________
% Header and Footers
%%%%%%%%%%%%%%%%%%%%%%%%%%%%%%%%%
\pagestyle{fancy}
\chead{}
\rhead{}
\lfoot{}
\rfoot{\footnotesize Page \thepage}
\lhead{}
%\rfoot{\footnotesize Page \thepage } % "e.g. Page 2"
\cfoot{}

%\setlength\headheight{30pt}
%%%%%%%%%%%%%%%%%%%%%%%%%%%%%%%%%
%________________________

\headsep 35pt % So that header does not go over title




\hypertarget{introduction}{%
\section{\texorpdfstring{Introduction
\label{Introduction}}{Introduction }}\label{introduction}}

References are to be made as follows:
\protect\hyperlink{ref-fama1997}{Fama \& French}
(\protect\hyperlink{ref-fama1997}{1997: 33}) and
\protect\hyperlink{ref-grinold2000}{Grinold \& Kahn}
(\protect\hyperlink{ref-grinold2000}{2000}) Such authors could also be
referenced in brackets (\protect\hyperlink{ref-grinold2000}{Grinold \&
Kahn, 2000}) and together \protect\hyperlink{ref-grinold2000}{Grinold \&
Kahn} (\protect\hyperlink{ref-grinold2000}{2000}). Source the reference
code from scholar.google.com by clicking on ``cite'\,' below article
name. Then select BibTeX at the bottom of the Cite window, and proceed
to copy and paste this code into your ref.bib file, located in the
directory's Tex folder. Open this file in Rstudio for ease of
management, else open it in your preferred Tex environment. Add and
manage your article details here for simplicity - once saved, it will
self-adjust in your paper.

\begin{quote}
I suggest renaming the top line after @article, as done in the template
ref.bib file, to something more intuitive for you to remember. Do not
change the rest of the code. Also, be mindful of the fact that bib
references from google scholar may at times be incorrect. Reference
Latex forums for correct bibtex notation.
\end{quote}

To reference a section, you have to set a label using
``\textbackslash label'\,' in R, and then reference it in-text as
e.g.~referencing a later section, Section \ref{Meth}.

Writing in Rmarkdown is surprizingly easy - see
\href{https://www.rstudio.com/wp-content/uploads/2015/03/rmarkdown-reference.pdf}{this
website} cheatsheet for a summary on writing Rmd writing tips.

\hypertarget{data}{%
\section*{Data}\label{data}}
\addcontentsline{toc}{section}{Data}

Notice how I used the curly brackets and dash to remove the numbering of
the data section.

Discussion of data should be thorough with a table of statistics and
ideally a figure.

In your tempalte folder, you will find a Data and a Code folder. In
order to keep your data files neat, store all of them in your Data
folder. Also, I strongly suggest keeping this Rmd file for writing and
executing commands, not writing out long pieces of data-wrangling. In
the example below, I simply create a ggplot template for scatter plot
consistency. I suggest keeping all your data in a data folder.

\hypertarget{literature-review}{%
\section{Literature review}\label{literature-review}}

The aim of this project is to determine whether U.S Presidential
speeches have predictive power over stock market movements. A positive
result would be powerful and could add to the ability of traders to
accurately predict stock market movements. Broadly, machine learning has
been selected as the method of modelling and the S\&P 500 index has been
chosen as representative of the `stock market'.

The undertaking of this project assumes 2 important conjectures. That
there is a correlation between the linguistic factors\footnote{`Linguistic
  factors' in this sense is intended to mean any and all patterns that
  can be detected in spoken language including verbiage, lexicon, tone,
  register, sentence length, word combination etc.} of speech employed
by U.S. presidents in their speeches and U.S. stock market movements;
and that the prominence of speech factors can be quantified by using
machine learning algorithms. The following literature review defends the
two conjectures, confirms the novelty of the project, surveys methods of
data cleaning that may be applicable and to discover which machine
learning methods are most appropriate for this project.

\hypertarget{conjecture-defence}{%
\subsection{Conjecture defence}\label{conjecture-defence}}

The following section references the findings of four peer-reviewed
articles in defence of the conjectures made in section 1. More evidence
exists and is reviewed in later sections but is not necessary to defend
the conjectures made here.

\hypertarget{fomc-speeches-and-u.s.-financial-market-reactions}{%
\subsubsection{FOMC speeches and U.S. financial market
reactions}\label{fomc-speeches-and-u.s.-financial-market-reactions}}

Hayo, Kutan \& Neuenkirch (2008) performed a generalized autoregressive
conditionally heteroscedastic (GARCH) analysis of the relationship
between Federal Open Market Committee (FOMC) speeches and the U.S.
financial market. The analysis was quantitative on the market side and
qualitative on the speech side. They found that FOMC speeches influence
trader behaviour, but that this effect is both asymmetrical (negative
impacts were larger than positive impacts) and non-uniform across trader
type (bond markets were affected far more than financial and forex
markets). Further, more formal modes of communication have a larger
impact on both returns and conditional variance and more prominent
speakers have a greater impact on bond markets. Finally, they found that
volatility in 3- and 6-month T-bills was reduced on the day of a speech.

It was commented that heteroskedasticity left something to be desired
when assessing the effects of monetary shocks. Further, it was found
that speeches alone were not sufficient to create significant effects
for financial markets. It is important that news agencies propagate the
news for market repercussions to occur. In a brief, informal interview
with a few bond traders it was discovered that they tended to ``read
monetary policy statements and listen to speeches by Greenspan
(Bernanke) themselves. Other types of communications are rather
neglected and the traders tend to rely on newswire information'' (Hayo
et al., 2008:27). Further, it was noted that news articles fail to take
a neutral stance on the contents of speeches implying that the market
effect may be distorted by the sentiment of news agencies.

This article shows that the sentiments of communications influence the
effect on markets. Thus, analysing sentiment is an important factor in
an accurate appraisal of the relationship between speeches and markets.
The finding that speeches need to be propagated by news agencies in
order to have significant impacts on financial markets is counter to the
hypothesis of this project but the FOMC is less publicly scrutinized
than the U.S. president which may render this finding inconsequential to
this project.

\hypertarget{political-speeches-and-stock-market-outcomes}{%
\subsubsection{Political speeches and stock market
outcomes}\label{political-speeches-and-stock-market-outcomes}}

Maligkris (2017) demonstrates that the speeches given by U.S.
presidential candidates directly influence the stock market,
particularly during the early months of their campaigns. These speeches
often contain information about potential presidents' positions on
policy changes and public issues. Thus, they can affect investor
sentiment and in turn the stock market. The employed methodology was to
analyse transcripts of presidential candidate speeches from the American
Presidency Project archives and the U.S. Government Publishing Project
from the 2004-2016 period according to the index developed in Baker,
Bloom \& Davis (2016) (explained in section 2.3). He shows, using
regression analysis that there is an increase in excess market returns
of 26 basis points following candidate speeches, however the direction
and magnitude of this effect varies between candidates. He goes on to
examine whether the difference in effect is due to heterogenous speech
content. Finally, it was demonstrated that speeches laden with economic
information tend to boost stock returns while also reducing volatility.
Speeches with a negative tone have the opposite effect. The long run
effect of speeches is dependent on market conditions.

This paper indicates that there is a correlation between presidential
candidates' speeches and stock movements. It is then reasonable to
believe that there is a correlation between presidential speeches and
stock market reactions. It also highlights that tone affects the
relationship and thus that the sentiment of a speech is important.
Further, the archives of the American Presidency Project and the U.S.
Government Publishing Project are good sources for U.S. political speech
transcripts -- the potential predictive data.

\hypertarget{measuring-economic-policy-uncertainty}{%
\subsubsection{Measuring economic policy
uncertainty}\label{measuring-economic-policy-uncertainty}}

Baker, Bloom \& Davis (2016) develop an Economic Policy Uncertainty
Index based on the frequency of articles containing a trio of terms in
10 leading U.S. newspapers. These terms were: ` ``economic'' or
``economy'' ; ``uncertain'' or ``uncertainty''; and one or more of
``congress'', ``deficit'', ``Federal Reserve'', `legislation'',
``regulation'', or ``White House''\,' (Baker et al., 2016:1594). The
terms were selected over a period of 24 months during which more than 15
000 news articles were read by humans in an auditing process. The index
proved to be quite accurate, spiking near the expected events, including
wars, tight presidential elections, fiscal policy battles and terrorist
activity.

They went on to show that their index had a strong relationship with
other economic uncertainty measures and policy uncertainty measures.
Further, congruence in uncertainty prediction was found between left-
and right- leaning newspapers.

This article shows that language processing can be used to predict
economic events, particularly economic uncertainty and economic policy
uncertainty. However, the type of language processing proposed for this
project differs significantly. While Baker et al.~(2016) used man hours
this project will use deep learning -- the result of which is likely to
be greater accuracy and reduced man hour expenditure, if it is correctly
applied.

\hypertarget{hope-change-and-financial-markets-can-obamas-words-drive-the-market}{%
\subsubsection{Hope, change and financial markets: can Obama's words
drive the
market?}\label{hope-change-and-financial-markets-can-obamas-words-drive-the-market}}

Sazedj \& Tavares (2011) asked whether the speeches made by former U.S.
President Barack Obama affected stock market prices. By regressing the
event of a speech during Obama's first 11 months in office on the daily
excess returns of the Dow Jones, the S\&P 500 and the NASDAQ they found
that the event of a speech had a generally insignificant effect on daily
excesses. However, by regressing key terms contained in 43 speeches
given during the first 11 months of Obama's presidency it was found that
the content of speeches can significantly affect daily excess returns
nearly uniformly across all three indices. Notably, the NASDAQ's
correlation to the content of speeches was weaker -- indicating that
technology markets may be less susceptible to presidential rhetoric.

This paper highlights the relationship between the content of a
presidential speech and stock market returns. This study was
correlational rather than predictive in nature.

\hypertarget{conjecture-defence-summary}{%
\subsubsection{Conjecture defence
summary}\label{conjecture-defence-summary}}

As seen in section 2.1.2. and section 2.1.4. it is true that there is a
correlation between the linguistic factors of speech employed by U.S.
presidents in their speeches and U.S. stock market movements (Sazedj \&
Tavares, 2011; Maligkris, 2017). Further, section 2.1. and section 2.3.
show that sentiment and other linguistic factors of speech can be
quantified using machine learning methods (Hayo, Kutan \& Neuenkirch,
2008; Baker, Bloom \& Davis, 2016). Thus, both conjectures hold and
further investigation is warranted. This is not the total of all
evidence supporting these conjectures but is sufficient. Other articles
reviewed here can be seen for further evidence.

\hypertarget{novelty}{%
\subsection{Novelty}\label{novelty}}

Searching `sentiment analysis stock market' on Google Scholar yields
mostly articles linking Twitter data and the stock market. Searching
`presidential speeches affect stock market' on Google Scholar yields
articles on the relationship between presidential speeches and the stock
market but none using Machine Learning techniques. Khedr, Salama \&
Yaseen (2017) describe related work as including relating news or
twitter data to stock market behaviour and prices and relating financial
news to stock prices, but do not mention presidential speeches.
Searching `machine learning S\&P 500' on Google Scholar yields an array
of articles using machine learning as a technical analysis tool but most
of these use other financial indicators and are not NLP based -- bar one
article analysing the effects of former U.S. president, Donald Trump's
tweets on the S\&P 500 and the DJIA. Searching `political speech machine
learning' on Google Scholar yields articles that focus on political
speeches but have no link to stock markets. Other searches yielded
similar results, thus as far as can be told -- this research is novel in
nature.

\hypertarget{data-cleaning-methods-for-nlp}{%
\subsection{Data cleaning methods for
NLP}\label{data-cleaning-methods-for-nlp}}

Katre (2019), in his analysis of Indian political speeches, uses Natural
language Toolkit (NLTK) and string methods to remove punctuation, HTML
tags and English stopwords\footnote{`Stopwords' are words that commonly
  occur across all speech and therefore only create noise in the data.
  Some examples are `the', `it', `they' and `and'.} , as well as
converting speeches to lowercase and tokenizing\footnote{`Tokenizing'
  refers to the splitting of words into tokens that have linguistic
  importance, for example the words `terrorism', `terrorist' and
  `terror' may all be tokenized to `terror'. Thus, the core concept of
  the word is captured while also simplifying the dataset.} them. Zubair
\& Cios (2015), before correlating the sentiment in Reuters articles
with S\&P 500 movements, clean their text data by tokenizing it with
NLTK. Kinyua et al.~(2021) clean their twitter data (tweets from then
U.S. president, Donald Trump) by deleting all tweets on days when the
stock trading was closed, deleting all tweets that only contained
standard stopwords and deleting all tweets that only contained URLs.
Khedr, Salama \& Yaseen (2017) tokenize, standardize by converting to
lowercase, remove stopwords from and stem\footnote{`Stemming' refers to
  the removal of suffixes to reduce the complexity of a dataset.} their
textual data before processing the abbreviations (replacing
abbreviations with the full phrase) and filtering out words that consist
of two or less characters.

\hypertarget{machine-learning-methods}{%
\subsection{Machine learning methods}\label{machine-learning-methods}}

The following section looks first at the literature informing the
sentiment analysis space and then at the literature around stock market
prediction in order to determine methods that suit the intersection
between the two.

\hypertarget{sentiment-analysis-methods}{%
\subsubsection{Sentiment analysis
methods}\label{sentiment-analysis-methods}}

Ren, Wu \& Liu (2019) analyse news articles at the sentence level by
assigning a sentiment polarity (using software designed for the Chinese
language) to each word followed by a sentiment score for each sentence
in a document. Each document is then categorized and a sentiment score
between -1 and 1 for all news for that day is generated. Zubair \& Cios
(2015) use the positive and negative valence categories from the Harvard
General Enquirer (HGI) to assign each word in a Reuters news article a
positive or negative label. They then sum the positives and negatives
into a tuple and divide that tuple by the number of words in an article
in order to create a vector that represents each news article. The
vectors are organized into time series, normalized by dividing all
vectors by the first vector, parsed through a Kalman filter and then
correlated to S\&P 500 returns using Pearson correlation (for both the
positive and negative scalar in the vector). Kinyua et al., (2021) use
the Valence Aware Dictionary for Sentiment Reasoning (VADER) to create a
sentiment feature for former U.S. president Donald Trump's tweets which
was then used as a regression feature in linear, decision tree and
random forest regressions. Khedr, Salama \& Yaseen (2017) use N-gram
(n=2) to extract key phrases from their corpus of news text data, then
term-frequency inverse-document-frequency is used to determine the
importance of those phrases within the corpus, and finally use a
naïve-Bayes classifier to assign positive and negative labels to each
news document. Purevdagva et al.~(2020) use a variety of features
present in both data and metadata to predict fake political speech. Two
features relevant to this project were `speaker job' and `context'
(press, direct or social) which were labelled using universal serial
encoders. For the actual sentiment analysis they used the linguistic
inquiry and word count (LIWC) tool to categorize and count words into
emotional, cognitive and structural text components. Various further
attempts to extract sentiment from the text did not yield increased
prediction accuracies. They go on to use an extra tree classifier for
feature selection and then support vector machine (SVM), multilayer
perceptron, convolutional neural network, decision trees, fasttext and
bidirectional encoder representations from transformers (BERT) for
prediction with highest accuracy coming from the SVM. Dilai, Onukevych
\& Dilay, (2018) use SentiStrength -- an automatic sentiment analysis
tool - to compare the sentiment in speeches between former U.S.
president Donald Trump and former Ukrainian president Petro Poroshenko.

\hypertarget{stock-market-prediction-methods}{%
\subsubsection{Stock market prediction
methods}\label{stock-market-prediction-methods}}

Ren, Wu \& Liu (2019) use an SVM and fivefold cross validation approach
to achieve a prediction accuracy of 98\% when predicting fake news in
political speech. They combined sentiment data and market indicators as
their input data. Kinyua et al.~(2021) use linear, decision tree and
random forest regressions to predict S\&P 500 and DJIA directional
changes. Random forest regression performed best for both datasets.
Khedr, Salama \& Yaseen (2017) use open, high, low and close prices from
their stock market data as features after labelling the change from the
previous day as positive, negative or equal. Jiao \& Jakubowicz (2017)
extracted lag and window features from the S\&P 500 and the global 8
index before running time series random forest, neural network and
gradient boosted trees to predict movements of individual stocks in the
S\&P 500. Liu et al.~(2016) used forward search feature selection to
select features for SVM, naïve-Bayes, Gaussian discriminant analysis and
logistic regression from a set of economic features including the crude
oil daily return, currency exchange rates and major stock indices daily
returns in order to forecast the S\&P 500 movement.

\hypertarget{summary-of-literature-review}{%
\subsection{Summary of literature
review}\label{summary-of-literature-review}}

Table 2.1 represents the literature review in a condensed format which
allows for easy comparison of the data and methods used and resulting
accuracies. Table 2.2 represents the metadata, linked through `Paper
number' for Table 2.1.

\begin{longtable}[]{@{}
  >{\centering\arraybackslash}p{(\columnwidth - 10\tabcolsep) * \real{0.07}}
  >{\centering\arraybackslash}p{(\columnwidth - 10\tabcolsep) * \real{0.26}}
  >{\centering\arraybackslash}p{(\columnwidth - 10\tabcolsep) * \real{0.31}}
  >{\centering\arraybackslash}p{(\columnwidth - 10\tabcolsep) * \real{0.17}}
  >{\centering\arraybackslash}p{(\columnwidth - 10\tabcolsep) * \real{0.13}}
  >{\centering\arraybackslash}p{(\columnwidth - 10\tabcolsep) * \real{0.06}}@{}}
\caption{Condensed literature review}\tabularnewline
\toprule
\begin{minipage}[b]{\linewidth}\centering
Paper number
\end{minipage} & \begin{minipage}[b]{\linewidth}\centering
ML Method
\end{minipage} & \begin{minipage}[b]{\linewidth}\centering
Cleaning method
\end{minipage} & \begin{minipage}[b]{\linewidth}\centering
Data type
\end{minipage} & \begin{minipage}[b]{\linewidth}\centering
Index predicted
\end{minipage} & \begin{minipage}[b]{\linewidth}\centering
Max ccuracy
\end{minipage} \\
\midrule
\endfirsthead
\toprule
\begin{minipage}[b]{\linewidth}\centering
Paper number
\end{minipage} & \begin{minipage}[b]{\linewidth}\centering
ML Method
\end{minipage} & \begin{minipage}[b]{\linewidth}\centering
Cleaning method
\end{minipage} & \begin{minipage}[b]{\linewidth}\centering
Data type
\end{minipage} & \begin{minipage}[b]{\linewidth}\centering
Index predicted
\end{minipage} & \begin{minipage}[b]{\linewidth}\centering
Max ccuracy
\end{minipage} \\
\midrule
\endhead
1 & Support vector machine with fivefold cross validation & & Daily
online stock reviews & SSE 50 & 0.98 \\
1 & Support vector machine with rolling windows & & & & 0.9 \\
1 & Logistic regression with fivefold cross validation & & & & 0.87 \\
2 & Non-ML: & Tokenized and mined using the Harvard General Inquirer
dictionary & Reuters textual data & S\&P 500 & Corr: -0.91 \\
3 & Random forest & Date validation, stop word and URL removal & Tweets
& INDU & 0.98 \\
3 & Decision tree & & & & 0.97 \\
3 & Logistic regression & & & & 0.81 \\
3 & Random forest & & & S\&P 500 & 0.92 \\
3 & Decision tree & & & & 0.88 \\
3 & Logistic regression & & & & 0.77 \\
4 & N-gram, TF-IDF, Naïve Bayes, K-NN & Tokenize, stopwords, stemming,
abbreviation processing & News articles and financial reports & Three
tech stocks & 0.9 \\
5 & TS logistic regression & Feature extraction: lags, window features &
Financial indicators & Individual S\&P 500 stocks & 0.79 \\
5 & TS random forest & & & & 0.78 \\
5 & TS neural network & & & & 0.78 \\
5 & TS gradient boosting & & & & 0.78 \\
6 & Logistic regression & Date validation forward search feature
selection & Market indices, exchange rates & S\&P 500 & 0.61 \\
6 & Gaussian discriminant analysis & & & & 0.61 \\
6 & Naïve Bayes & & & & 0.6 \\
6 & Linear SVM & & & & 0.6 \\
6 & Radial Basis Function SVM & & & & 0.63 \\
6 & Polynomial SVM & & & & 0.6 \\
7 & SVM & Feature extraction and feature selection & Political speeches
and metadata & Liar dataset & 0.74 \\
7 & Multilayer perceptron & & & & 0.55 \\
7 & Convolutional neural network & & & & 0.61 \\
7 & Fasttext & & & & 0.66 \\
7 & BERT & & & & 0.66 \\
\bottomrule
\end{longtable}

\begin{longtable}[]{@{}
  >{\centering\arraybackslash}p{(\columnwidth - 6\tabcolsep) * \real{0.07}}
  >{\centering\arraybackslash}p{(\columnwidth - 6\tabcolsep) * \real{0.61}}
  >{\centering\arraybackslash}p{(\columnwidth - 6\tabcolsep) * \real{0.16}}
  >{\centering\arraybackslash}p{(\columnwidth - 6\tabcolsep) * \real{0.17}}@{}}
\caption{Table 2.1 metadata}\tabularnewline
\toprule
\begin{minipage}[b]{\linewidth}\centering
Paper number
\end{minipage} & \begin{minipage}[b]{\linewidth}\centering
Title
\end{minipage} & \begin{minipage}[b]{\linewidth}\centering
Citation
\end{minipage} & \begin{minipage}[b]{\linewidth}\centering
DOI
\end{minipage} \\
\midrule
\endfirsthead
\toprule
\begin{minipage}[b]{\linewidth}\centering
Paper number
\end{minipage} & \begin{minipage}[b]{\linewidth}\centering
Title
\end{minipage} & \begin{minipage}[b]{\linewidth}\centering
Citation
\end{minipage} & \begin{minipage}[b]{\linewidth}\centering
DOI
\end{minipage} \\
\midrule
\endhead
1 & Forecasting Stock Market Movement Direction Using Sentiment Analysis
and Support Vector Machine & (Ren, Wu \& Liu, 2019) &
10.1109/JSYST.2018.2794462 \\
2 & Extracting News Sentiment and Establishing its Relationship with the
S\&P 500 Index & (Zubair \& Cios, 2015) & 10.1109/JSYST.2018.2794462 \\
3 & An analysis of the impact of President Trump's tweets on the DJIA
and S\&P 500 using machine learning and sentiment analysis & (Kinyua et
al., 2021) & 10.1016/j.jbef.2020.100447 \\
4 & Predicting Stock Market Behaviour using Data Mining Technique and
News Sentiment Analysis & (Khedr, Salama \& Yaseen, 2017) &
10.5815/ijisa.2017.07.03 \\
5 & Predicting Stock Movement Direction with Machine Learning:an
Extensive Study on S\&P 500 Stocks & (Jiao \& Jakubowicz, 2017 &
10.1109/BigData.2017.8258518 \\
6 & Forecasting S\&P 500 Stock Index Using Statistical Learning Models &
(Liu et al., 2016) & 10.4236/ojs.2016.66086 \\
7 & A machine-learning based framework for detection of fake political
speech & (Purevdagva et al., 2020) &
10.1109/BigDataSE50710.2020.00019 \\
\bottomrule
\end{longtable}

\hypertarget{data-collection-and-cleaning-process}{%
\section{Data collection and cleaning
process}\label{data-collection-and-cleaning-process}}

The following is a description of the data collection and cleaning
process and the feature extraction process. It begins by explaining the
three datasets collected for this study, namely control, meta and test
data. Control refers to autoregressive features extracted from the S\&P
500, meta refers to S\&P 500 adjacent financial data and test refers to
vectorised presidential speeches which are the focus of this study. The
section begins by elaborating on the collection and cleaning steps for
each of these datasets. The next subsection describes the feature
engineering methods (sentiment analysis and word vectorization) that
were used to draw variables with potentially strong signals from the
text data. The final subsection describes the reasoning, methods and
results used in a time series analysis of the financial time series
(S\&P 500) data in order to extract useable autoregressive features from
it.

\hypertarget{data-1}{%
\subsection{Data}\label{data-1}}

Three types of data were gathered for this project, namely: presidential
speeches/text data (all the transcripts from the Presidency Project
website including formal and informal and written and verbal addresses),
a history of the S\&P 500 index and a history of 6 S\&P 500 adjacent
price histories -- 2 sets of financial data.

The S\&P 500 index and the metadata was downloaded from Yahoo Finance
while the presidential speeches were scraped from the American
Presidency Project (Yahoo Finance, n.d.; Woolley \& Peters, n.d.). The
S\&P 500 is downloaded using the `fin\_data\_downloader' Python module
and will always download the entire history of the S\&P 500 index at a
daily interval. The same goes for the metadata. The presidential
speeches on the other hand were scraped using the `WebScrapeAndClean'
Python module which will also always scrape the entire corpus available
on the American Presidency Project website. This allows for perfectly
updated data to be collected at any time.

The S\&P 500 data contains seven variables, namely: `Date', `Open',
`High', `Low', `Close', `Adj Close', and `Volume'. `Open' records the
opening price for each day, while `Close' records the closing price.
`High' records the daily high and `Low' the daily low. `Volume' records
the dollar amount of stock traded in the S\&P 500 on each day and `Adj
Close' is irrelevant because it never differs from `Close'. Notably,
opening and closing prices are only differentiated between after April
20th 1982 while volume was only recorded after 1950.

After scraping, the Presidency Project data contains five variables.
These are `Type' which records the category and sub-category of each
speech, `Name' which records the title and name of the main speaker,
`Date' which records the date the speech occurred on, `Title' which
records the title of each speech and `Transcript' which contains the raw
HTML transcript of the speech.

\hypertarget{cleaning}{%
\subsection{Cleaning}\label{cleaning}}

The S\&P 500 index did not require any cleaning after download, besides
the removal of the redundant `Adj Close' variable. Conversely, the
presidential speeches required extensive cleaning. Text that has been
web-scraped contains HTML tags\footnote{HTML tags include paragraphing
  and spacing indicators for computers to interpret such as '} , thus
the first step was to remove the HTML tags from the text. Similarly, the
test contained reactions from the crowds listening to the
speeches\footnote{These were tags such as `Laughter' and `Applause'.} ,
which were also removed. Next, the transcripts were converted to lower
case and the question sections removed. Next a `No Stops Transcript'
variable was created by removing the stopwords\footnote{Stopwords are
  words that commonly occur in the English language and are therefore
  unlikely to contain any sort of signal and thus constitute only noise.}
in the Natural Language Tool Kit (NLTK) stop words dictionary from the
clean transcript. The original transcripts were also kept. Table A1 in
Appendix A shows the shape of the data at this point. The cleaning of
both the speech data and both financial datasets was done in their
original collection scripts, i.e.~the `fin\_data\_downloader' Python
script and the `WebScrapeAndClean' Python script respectively.

\hypertarget{text-feature-engineering}{%
\subsection{Text feature engineering}\label{text-feature-engineering}}

To increase the strength of the signals coming out of the speech data
and reduce the computational power required to run the machine learning
algorithms -- feature engineering is required. Feature engineering
refers to the emphasis of certain signals within the available data and
creation of new variables which capture these signals. It results in the
addition of extra features (variables) to the dataset. There are three
broad methods of feature engineering: feature selection, feature
extraction and the creation of new features (Géron, 2019:27). In this
section the creation of new features occurred and took the form of
sentiment analysis and word vectorization for the text data.

\hypertarget{sentiment-analysis}{%
\subsubsection{Sentiment analysis}\label{sentiment-analysis}}

Two versions of sentiment analysis were carried out. First, NLTK's
Valence Aware Dictionary for Sentiment Reasoning (VADER) was used to
extract a sentiment analysis tuple, in this instance containing four
scores, namely, negativity, neutrality, positivity and compound. VADER
is a lexicon based system of sentiment analysis (Sohangir, Petty \&
Wang, 2018). Each of negativity, neutrality and positivity describe a
transcript independently of the other scores while compound describes a
transcript comprehensively (combining the other three scores). VADER,
when compared with alternative NLP feature extraction techniques has
performed better on social media transcripts and generalized better to
other areas (Hutto \& Gilbert, 2014; Elbagir \& Yang, 2019). VADER has
been used in vectorizing text relative to financial data in Pano \&
Kashef (2020) for Bitcoin price predictions, Agarwal (2020) which found
a strong correlation between VADER sentiment scores and stock price
changes and Sohangir et al.~(2018) which shows the superiority of
lexicon based approaches (specifically VADER) over ML approaches for
sentiment classification.

Next, the TextBlob package's sentiment analysis tool was used. This
yields two scores (in a tuple) describing the sentiment of a transcript,
namely, a polarity score (ranging from -1 to 1) and a subjectivity score
(ranging from 0 to 1). Polarity describes whether the emotions expressed
are negative or positive, with lower scores indicating negativity while
subjectivity indicates the extent of the usage of subjective words
(Chudy, n.d.). Biswas, Sarkar, Das, Bose \& Roy (2020) use Textblob
sentiment scores in their analysis of the effects of Covid-19 on stock
markets. Textblob was also tested in Sohangir et al.~(2018) but did not
perform as well as VADER although it did outperform ML methods in terms
of area under the curve (AUC) scores. It is expected that the VADER
sentiment tuple will outperform Textblob's sentiment tuple as a
predictor of the S\&P 500 data.

\hypertarget{speech-vectorization}{%
\subsubsection{Speech vectorization}\label{speech-vectorization}}

Converting human readable text into machine readable data requires the
conversion from words to numbers. This vectorization can be done in
various ways but in order to preserve the meaning of the texts the
Word2Vec and Doc2Vec Python packages provided by Gensim were used (Řeh
uřek \& Sojka, 2010). However, Word2Vec was originally published in two
papers by Mikolov, Chen, Corrado \& Dean (2013a,b) while Doc2Vec was
suggested by Le \& Mikolov (2014).

\hypertarget{word2vec}{%
\paragraph{Word2Vec}\label{word2vec}}

Gensim's Word2Vec Python package's skip-gram model is used for Word2Vec
vectorization. Using the full vocabulary of words in a corpus of
speeches and one-hot encoding for word vectors, Word2Vec trains a single
hidden layer neural network to predict words based on the words around
them. The parameters used for training are available in Appendix A.

The model contextually embeds each word in the entire corpus of speeches
by running the speeches through the single hidden layer predictive
neural network (NN). The NN is provided with the pseudo-task of
predicting the word of focus from the words surrounding it each time it
occurs in the corpus. Thus, a hidden layer is trained to contain the
information that contextually embeds each word in the corpus. These
hidden layer vectors (rather than the predictions) are the real output
of the Word2Vec model. Words that appear in similar contexts throughout
the corpus will have similar representational vectors (hidden layers)
and thus can be said to have similar meanings in the corpus(Mikolov et
al., 2013a,b).

An example phrase might be `The quick brown fox jumps'. If the word
`brown' is the focus word, the words `quick' and `fox' would be fed into
the neural network which would then be trained to map the input to the
word `brown'. Doing this for every instance of `brown' in a corpus
creates a hidden layer that contains all the contextual information
required to predict the word `brown' in the given corpus. Figure 1
depicts this process.

The model is extremely good at relating words that appear in similar
contexts to each other. For example, when asked for the three words most
similar to `oil' the model trained on the presidential speeches corpus
returns `crude', `gas' and `petroleum'. The input `gold' returns
`silver', `bullion' and `coin'; whilst `virus' returns `covid19',
`infection' and `pandemic'. In this study the representational vectors
each contain 200 elements (because the hidden layers were set to contain
200 elements). Thus each word is described by a vector containing 200
elements. In order to create a similarly sized vector for each speech,
the vectors describing all the words in each speech were averaged. Thus
each speech has been reduced to a 200 element vector averaging the
contextual embeddings of each word contained therein. This averaging
technique was also used in Vargas, de Lima \& Evsukoff (2017) and Qin \&
Ji (2018). However, this method fails to preserve word order in a
document vector (Le \& Mikolov, 2014).

Notably, the algorithm also makes room for phrases such as `asset
backed' or `short selling' -- which are considered as `assetbacked' and
`shortselling'. When two words that occur irregularly in the vocabulary
occur together frequently the algorithm interprets them as a phrase and
treats them as such. There is, however, only room for two words in each
phrase if the algorithm has only been executed once -- which is the case
here.

Word2vec is used by (Shi, Zhao \& Xu, 2019) for the improvement of
sentiment classification which implies that the final vectors in this
study may contain sentiment information. It is also used by Vargas et
al.~(2017) and Qin \& Ji (2018)in their predictive modelling of S\&P 500
changes based on twitter data. Vargas et al.~(2017) also used 200
element vectors while Qin \& Ji (2018) used 300 element vectors. Both
studies achieved prediction accuracy around 65\%.

\begin{figure}
\centering
\includegraphics{Images/Word2Vec_model.png}
\caption{Word2Vec model}
\end{figure}

\hypertarget{doc2vec}{%
\paragraph{Doc2Vec}\label{doc2vec}}

An alternative method to creating a vector representation of a sentence
or document is Doc2Vec. This method is similar to the Word2Vec method
described above but includes an additional floating vector when
performing its pseudo-task. This vector is maintained across every word
prediction task within a document and subjected to training for each
instance of every word. Thus each document in a corpus is assigned a
single comprehensive vector that embeds it in the corpus. Embedding
within the broader corpus is maintained by tagging each document with a
unique tagging phrase. This method outperforms other methods such as
bag-of-words for text representations (Le \& Mikolov, 2014).

There are two implementations of Doc2Vec, namely Distributed Bag of
Words (DBOW) and Distributed Memory (DM) (Sohangir et al., 2018). Both
have been used in this study. In both cases each document is assigned a
paragraph tag which represents the paragraph to the Doc2Vec algorithm.
DM Doc2vec does this in the same way that a word represents itself to
the Word2Vec algorithm. For every word in a document, its paragraph tag
is passed to the Doc2Vec algorithm along with the words relevant to the
current prediction pseudo-task. Back propagation is employed in the same
manner as in Word2Vec except that the document tag vector is optimized
for separately from the word vectors. This document tag vector is the
relevant output in this case. Because a single vector of weights is
created for each document, this vector should constitute a vectorized
representation of the document as a whole. Figure 2 depicts the
architecture of DM Doc2Vec algorithms. Alternatively, DBOW Doc2Vec
algorithms ignore the context of a word and use random sampling to
predict words from a paragraph. Figure 3 depicts the architecture of a
DBOW Doc2Vec algorithm.

\begin{figure}
\centering
\includegraphics{Images/Doc2Vec_distributed_memory.png}
\caption{Doc2Vec model - distributed memory architecture: dm = 1}
\end{figure}

\begin{figure}
\centering
\includegraphics{Images/Doc2Vec_distributed_bow.png}
\caption{Doc2Vec model - distributed bag of words architecture: dm = 0}
\end{figure}

!!!!!!!!WHERE HAS DOC2VEC BEEN USEFUL FOR FINANCIAL
PREDICTIONS!!!!!!!!!!

\hypertarget{sp-500-time-series-analysis}{%
\subsection{S\&P 500 time series
analysis}\label{sp-500-time-series-analysis}}

As part of feature extraction - econometric time series analysis has
been run on the S\&P 500 data. The aim of this analysis was to find the
linear model of best fit to the S\&P 500 data and then include relevant
autoregressive variables in the final dataset under the assumption that
they will be relevant in the highly non-linear ML models. An initial
analysis was done on the S\&P 500 data after April 20th 1982 because
opening and closing prices are differentiated between from that date
onwards. This analysis found a large array of significant variables --
more than half the days of the month, the month of September, a few
specific years, 10 non-consecutive lags and the first lag of volume of
trade. The significance of these variables, particularly the days of the
month were difficult to explain rationally. However, they did indicate
persistence of volatility. This volatility persistence and the fact that
volume is a strong indicator of absolute price change encouraged a
second round of analysis that was done on all data following the initial
recording of volume in 1950. This second round of analysis was focussed
on finding an autoregressive conditional heteroskedasticity (ARCH) model
that fit the data well. The analysis concluded with the selection of an
ARMA (1,0) GARCH (1,1) model. Thus, the core variables selected for
inclusion in the final data set are the first lag of standardized
volume, the first lag of daily percentage change, the first residual of
daily percentage change (DPC) predicted by an ARMA (1,0) model and the
first lag of variance of the DPC.

\hypertarget{onwards}{%
\subsubsection{1982 onwards}\label{onwards}}

Running time series analysis on the daily percentage change in the S\&P
500 after April 20th 1982 has revealed 10 significant non-consecutive
lags. The `Daily percentage change `variable was created by taking the
percentage difference between the `Close' and `Open' variables for each
day of the data. Before April 20th 1982 the `Open' and `Close' variables
hold identical values, hence the time series analysis only being run
after that date. As can be seen in Figure 1, `Daily percentage change'
is naturally stationary. This is supported by the results of an
Augmented Dickey-Fuller unit root test which indicated that no unit root
is present in the data. Running a partial autocorrelation function
revealed 10 significant lags (these were lagged by 1, 2, 4, 12, 15, 16,
18, 27, 32, and 34 periods). Regressing `Daily percentage change' on all
10 significant lags reinforced the finding by yielding significance
above the 95\% confidence interval for all 10 lags. Further, regressing
the daily percentage change on weekday, monthday, month and year
categorical variables yielded the significant correlations depicted in
Table 1 (none of the weekdays were significant). Regressing on the
categorical variables and the lags simultaneously yielded similar
results with increased significance for 2002 (to the 99\% level) and
2008 (to the 99,9\% level) and the addition of 2018 (significant at the
95\% level), further an extra 4 monthdays were deemed significant -
bringing the total to 21 (out of 31) significant monthdays, finally,
some of the significance levels on the lags changed. These statistics
are depicted in Table 2. Adding a normalized (distributed standard
normal) volume variable lagged by one period to the regression yields a
significance at the 99\% level on the lagged volume variable and alters
the significance on the year variables as reported in Table 3.

While the years 2002 and 2008 are justifiable as significant because of
the financial crashes that happened in each of those years (2002 -- dot
com bubble and 2008 housing bubble) and September is also relatable to
the housing bubble; it is difficult to rationally justify the monthday
variables as significant regardless of their quantitative significance.
The high number of lags is also difficult to justify and implies rather
that there may be persistence of volatility. Thus the following section
focusses on the modelling of volatility over a time period that
maximises the inclusion of volume statistics in the data. All of the
analysis reported in this section was done in R using the packages
`stats', `dplyr', `urca', `tidyverse', `ggplot2' and `fixest' (Marais,
2022).

\begin{figure}[H]

{\centering \includegraphics{Thesis_WriteUp_files/figure-latex/Figure1-1} 

}

\caption{Caption Here \label{Figure1}}\label{fig:Figure1}
\end{figure}

To make your graphs look extra nice in latex world, you could use Tikz
device. Replace dev - `png' with `tikz' in the chunk below. Notice this
makes the build time longer and produces extra tex files - so if you are
comfortable with this, set your device to Tikz and try it out:

\begin{figure}[H]

{\centering \includegraphics{Thesis_WriteUp_files/figure-latex/Figure2-1} 

}

\caption{Caption Here \label{Figure2}}\label{fig:Figure2}
\end{figure}

To reference the plot above, add a ``\textbackslash label'\,' after the
caption in the chunk heading, as done above. Then reference the plot as
such: As can be seen, Figures \ref{Figure1} and \ref{Figure2} are
excellent, with Figure \ref{Figure2} being particularly aesthetically
pleasing due to its device setting of Tikz. The nice thing now is that
it correctly numbers all your figures (and sections or tables) and will
update if it moves. The links are also dynamic.

I very strongly suggest using ggplot2 (ideally in combination with
dplyr) using the ggtheme package to change the themes of your figures.

Also note the information that I have placed above the chunks in the
code chunks for the figures. You can edit any of these easily - visit
the Rmarkdown webpage for more information.

\hypertarget{splitting-a-page}{%
\section{Splitting a page}\label{splitting-a-page}}

You can also very easily split a page using built-in Pandoc formatting.
I comment this out in the code (as this has caused issues building the
pdf for some users - which I presume to be a Pandoc issue), but you are
welcome to try it out yourself by commenting out the following section
in your Rmd file.

\hypertarget{methodology}{%
\section{\texorpdfstring{Methodology
\label{Meth}}{Methodology }}\label{methodology}}

\hypertarget{subsection}{%
\subsection{Subsection}\label{subsection}}

Ideally do not overuse subsections. It equates to bad
writing.\footnote{This is an example of a footnote by the way. Something
  that should also not be overused.}

\hypertarget{math-section}{%
\subsection{Math section}\label{math-section}}

Equations should be written as such:

\begin{align}
\beta = \sum_{i = 1}^{\infty}\frac{\alpha^2}{\sigma_{t-1}^2} \label{eq1} \\
\int_{x = 1}^{\infty}x_{i} = 1 \notag
\end{align}

If you would like to see the equations as you type in Rmarkdown, use \$
symbols instead (see this for yourself by adjusted the equation):

\[
\beta = \sum_{i = 1}^{\infty}\frac{\alpha^2}{\sigma_{t-1}^2} \\
\int_{x = 1}^{\infty}x_{i} = 1
\]

Note again the reference to equation \ref{eq1}. Writing nice math
requires practice. Note I used a forward slashes to make a space in the
equations. I can also align equations using \textbf{\&}, and set to
numbering only the first line. Now I will have to type ``begin
equation'\,' which is a native \LaTeX command. Here follows a more
complicated equation:

\begin{align}
    y_t &= c + B(L) y_{t-1} + e_t   \label{eq2}    \\ \notag
    e_t &= H_t^{1/2}  z_t ; \quad z_t \sim  N(0,I_N) \quad \& \quad H_t = D_tR_tD_t \\ \notag
        D_t^2 &= {\sigma_{1,t}, \dots, \sigma_{N,t}}   \\ \notag
        \sigma_{i,t}^2 &= \gamma_i+\kappa_{i,t}  v_{i, t-1}^2 +\eta_i  \sigma_{i, t-1}^2, \quad \forall i \\ \notag
        R_{t, i, j} &= {diag(Q_{t, i, j}}^{-1}) . Q_{t, i, j} . diag(Q_{t, i, j}^{-1})  \\ \notag
        Q_{t, i, j} &= (1-\alpha-\beta)  \bar{Q} + \alpha  z_t  z_t'  + \beta  Q_{t, i, j} \notag
\end{align}

Note that in \ref{eq2} I have aligned the equations by the equal signs.
I also want only one tag, and I create spaces using ``quads'\,'.

See if you can figure out how to do complex math using the two examples
provided in \ref{eq1} and \ref{eq2}.

\hypertarget{results}{%
\section{Results}\label{results}}

Tables can be included as follows. Use the \emph{xtable} (or kable)
package for tables. Table placement = H implies Latex tries to place the
table Here, and not on a new page (there are, however, very many ways to
skin this cat. Luckily there are many forums online!).

\begin{table}[H]
\centering
\begin{tabular}{rrrrrrrrrrrr}
  \hline
 & mpg & cyl & disp & hp & drat & wt & qsec & vs & am & gear & carb \\ 
  \hline
1 & 21.00 & 6.00 & 160.00 & 110.00 & 3.90 & 2.62 & 16.46 & 0.00 & 1.00 & 4.00 & 4.00 \\ 
  2 & 21.00 & 6.00 & 160.00 & 110.00 & 3.90 & 2.88 & 17.02 & 0.00 & 1.00 & 4.00 & 4.00 \\ 
  3 & 22.80 & 4.00 & 108.00 & 93.00 & 3.85 & 2.32 & 18.61 & 1.00 & 1.00 & 4.00 & 1.00 \\ 
  4 & 21.40 & 6.00 & 258.00 & 110.00 & 3.08 & 3.21 & 19.44 & 1.00 & 0.00 & 3.00 & 1.00 \\ 
  5 & 18.70 & 8.00 & 360.00 & 175.00 & 3.15 & 3.44 & 17.02 & 0.00 & 0.00 & 3.00 & 2.00 \\ 
   \hline
\end{tabular}
\caption{Short Table Example \label{tab1}} 
\end{table}

To reference calculations \textbf{in text}, \emph{do this:} From table
\ref{tab1} we see the average value of mpg is 20.98.

Including tables that span across pages, use the following (note that I
add below the table: ``continue on the next page'\,'). This is a neat
way of splitting your table across a page.

Use the following default settings to build your own possibly long
tables. Note that the following will fit on one page if it can, but
cleanly spreads over multiple pages:

\begingroup\fontsize{12pt}{13pt}\selectfont
\begin{longtable}{rrrrrrrrrrr}
\caption{Long Table Example} \\ 
  \toprule
mpg & cyl & disp & hp & drat & wt & qsec & vs & am & gear & carb \\ 
  \hline 
\endhead 
\hline 
{\footnotesize Continued on next page} 
\endfoot 
\endlastfoot 
 \midrule
21.00 & 6.00 & 160.00 & 110.00 & 3.90 & 2.62 & 16.46 & 0.00 & 1.00 & 4.00 & 4.00 \\ 
  21.00 & 6.00 & 160.00 & 110.00 & 3.90 & 2.88 & 17.02 & 0.00 & 1.00 & 4.00 & 4.00 \\ 
  22.80 & 4.00 & 108.00 & 93.00 & 3.85 & 2.32 & 18.61 & 1.00 & 1.00 & 4.00 & 1.00 \\ 
  21.40 & 6.00 & 258.00 & 110.00 & 3.08 & 3.21 & 19.44 & 1.00 & 0.00 & 3.00 & 1.00 \\ 
  18.70 & 8.00 & 360.00 & 175.00 & 3.15 & 3.44 & 17.02 & 0.00 & 0.00 & 3.00 & 2.00 \\ 
  18.10 & 6.00 & 225.00 & 105.00 & 2.76 & 3.46 & 20.22 & 1.00 & 0.00 & 3.00 & 1.00 \\ 
  14.30 & 8.00 & 360.00 & 245.00 & 3.21 & 3.57 & 15.84 & 0.00 & 0.00 & 3.00 & 4.00 \\ 
  24.40 & 4.00 & 146.70 & 62.00 & 3.69 & 3.19 & 20.00 & 1.00 & 0.00 & 4.00 & 2.00 \\ 
  22.80 & 4.00 & 140.80 & 95.00 & 3.92 & 3.15 & 22.90 & 1.00 & 0.00 & 4.00 & 2.00 \\ 
  19.20 & 6.00 & 167.60 & 123.00 & 3.92 & 3.44 & 18.30 & 1.00 & 0.00 & 4.00 & 4.00 \\ 
  17.80 & 6.00 & 167.60 & 123.00 & 3.92 & 3.44 & 18.90 & 1.00 & 0.00 & 4.00 & 4.00 \\ 
  16.40 & 8.00 & 275.80 & 180.00 & 3.07 & 4.07 & 17.40 & 0.00 & 0.00 & 3.00 & 3.00 \\ 
  17.30 & 8.00 & 275.80 & 180.00 & 3.07 & 3.73 & 17.60 & 0.00 & 0.00 & 3.00 & 3.00 \\ 
  15.20 & 8.00 & 275.80 & 180.00 & 3.07 & 3.78 & 18.00 & 0.00 & 0.00 & 3.00 & 3.00 \\ 
  10.40 & 8.00 & 472.00 & 205.00 & 2.93 & 5.25 & 17.98 & 0.00 & 0.00 & 3.00 & 4.00 \\ 
  10.40 & 8.00 & 460.00 & 215.00 & 3.00 & 5.42 & 17.82 & 0.00 & 0.00 & 3.00 & 4.00 \\ 
  14.70 & 8.00 & 440.00 & 230.00 & 3.23 & 5.34 & 17.42 & 0.00 & 0.00 & 3.00 & 4.00 \\ 
  32.40 & 4.00 & 78.70 & 66.00 & 4.08 & 2.20 & 19.47 & 1.00 & 1.00 & 4.00 & 1.00 \\ 
  30.40 & 4.00 & 75.70 & 52.00 & 4.93 & 1.61 & 18.52 & 1.00 & 1.00 & 4.00 & 2.00 \\ 
  33.90 & 4.00 & 71.10 & 65.00 & 4.22 & 1.83 & 19.90 & 1.00 & 1.00 & 4.00 & 1.00 \\ 
  21.50 & 4.00 & 120.10 & 97.00 & 3.70 & 2.46 & 20.01 & 1.00 & 0.00 & 3.00 & 1.00 \\ 
  15.50 & 8.00 & 318.00 & 150.00 & 2.76 & 3.52 & 16.87 & 0.00 & 0.00 & 3.00 & 2.00 \\ 
  15.20 & 8.00 & 304.00 & 150.00 & 3.15 & 3.44 & 17.30 & 0.00 & 0.00 & 3.00 & 2.00 \\ 
  13.30 & 8.00 & 350.00 & 245.00 & 3.73 & 3.84 & 15.41 & 0.00 & 0.00 & 3.00 & 4.00 \\ 
  19.20 & 8.00 & 400.00 & 175.00 & 3.08 & 3.85 & 17.05 & 0.00 & 0.00 & 3.00 & 2.00 \\ 
  27.30 & 4.00 & 79.00 & 66.00 & 4.08 & 1.94 & 18.90 & 1.00 & 1.00 & 4.00 & 1.00 \\ 
  26.00 & 4.00 & 120.30 & 91.00 & 4.43 & 2.14 & 16.70 & 0.00 & 1.00 & 5.00 & 2.00 \\ 
  30.40 & 4.00 & 95.10 & 113.00 & 3.77 & 1.51 & 16.90 & 1.00 & 1.00 & 5.00 & 2.00 \\ 
  15.80 & 8.00 & 351.00 & 264.00 & 4.22 & 3.17 & 14.50 & 0.00 & 1.00 & 5.00 & 4.00 \\ 
  19.70 & 6.00 & 145.00 & 175.00 & 3.62 & 2.77 & 15.50 & 0.00 & 1.00 & 5.00 & 6.00 \\ 
  15.00 & 8.00 & 301.00 & 335.00 & 3.54 & 3.57 & 14.60 & 0.00 & 1.00 & 5.00 & 8.00 \\ 
  21.40 & 4.00 & 121.00 & 109.00 & 4.11 & 2.78 & 18.60 & 1.00 & 1.00 & 4.00 & 2.00 \\ 
   \bottomrule
\end{longtable}
\endgroup

\hfill

\hypertarget{huxtable}{%
\subsection{Huxtable}\label{huxtable}}

Huxtable is a very nice package for making working with tables between
Rmarkdown and Tex easier.

This cost some adjustment to the Tex templates to make it work, but it
now works nicely.

See documentation for this package
\href{https://hughjonesd.github.io/huxtable/huxtable.html}{here}. A
particularly nice addition of this package is for making the printing of
regression results a joy (see
\href{https://hughjonesd.github.io/huxtable/huxtable.html\#creating-a-regression-table}{here}).
Here follows an example:

If you are eager to use huxtable, comment out the Huxtable table in the
Rmd template, and uncomment the colortbl package in your Rmd's root.

Note that I do not include this in the ordinary template, as some latex
users have complained it breaks when they build their Rmds (especially
those using tidytex - I don't have this problem as I have the full
Miktex installed on mine). Up to you, but I strongly recommend
installing the package manually and using huxtable. To make this work,
uncomment the \emph{Adding additional latex packages} part in yaml at
the top of the Rmd file. Then comment out the huxtable example in the
template below this line. Reknit, and enjoy.

 
  \providecommand{\huxb}[2]{\arrayrulecolor[RGB]{#1}\global\arrayrulewidth=#2pt}
  \providecommand{\huxvb}[2]{\color[RGB]{#1}\vrule width #2pt}
  \providecommand{\huxtpad}[1]{\rule{0pt}{#1}}
  \providecommand{\huxbpad}[1]{\rule[-#1]{0pt}{#1}}

\begin{table}[ht]
\begin{centerbox}
\begin{threeparttable}
\captionsetup{justification=centering,singlelinecheck=off}
\caption{Regression Output}
 \label{Reg01}
\setlength{\tabcolsep}{0pt}
\begin{tabular}{l l l l}


\hhline{>{\huxb{0, 0, 0}{0.8}}->{\huxb{0, 0, 0}{0.8}}->{\huxb{0, 0, 0}{0.8}}->{\huxb{0, 0, 0}{0.8}}-}
\arrayrulecolor{black}

\multicolumn{1}{!{\huxvb{0, 0, 0}{0}}c!{\huxvb{0, 0, 0}{0}}}{\huxtpad{6pt + 1em}\centering \hspace{6pt} {\fontsize{12pt}{14.4pt}\selectfont } \hspace{6pt}\huxbpad{6pt}} &
\multicolumn{1}{c!{\huxvb{0, 0, 0}{0}}}{\huxtpad{6pt + 1em}\centering \hspace{6pt} {\fontsize{12pt}{14.4pt}\selectfont Reg1} \hspace{6pt}\huxbpad{6pt}} &
\multicolumn{1}{c!{\huxvb{0, 0, 0}{0}}}{\huxtpad{6pt + 1em}\centering \hspace{6pt} {\fontsize{12pt}{14.4pt}\selectfont Reg2} \hspace{6pt}\huxbpad{6pt}} &
\multicolumn{1}{c!{\huxvb{0, 0, 0}{0}}}{\huxtpad{6pt + 1em}\centering \hspace{6pt} {\fontsize{12pt}{14.4pt}\selectfont Reg3} \hspace{6pt}\huxbpad{6pt}} \tabularnewline[-0.5pt]


\hhline{>{\huxb{255, 255, 255}{0.4}}->{\huxb{0, 0, 0}{0.4}}->{\huxb{0, 0, 0}{0.4}}->{\huxb{0, 0, 0}{0.4}}-}
\arrayrulecolor{black}

\multicolumn{1}{!{\huxvb{0, 0, 0}{0}}l!{\huxvb{0, 0, 0}{0}}}{\huxtpad{6pt + 1em}\raggedright \hspace{6pt} {\fontsize{12pt}{14.4pt}\selectfont (Intercept)} \hspace{6pt}\huxbpad{6pt}} &
\multicolumn{1}{r!{\huxvb{0, 0, 0}{0}}}{\huxtpad{6pt + 1em}\raggedleft \hspace{6pt} {\fontsize{12pt}{14.4pt}\selectfont -2256.361 ***} \hspace{6pt}\huxbpad{6pt}} &
\multicolumn{1}{r!{\huxvb{0, 0, 0}{0}}}{\huxtpad{6pt + 1em}\raggedleft \hspace{6pt} {\fontsize{12pt}{14.4pt}\selectfont 5763.668 ***} \hspace{6pt}\huxbpad{6pt}} &
\multicolumn{1}{r!{\huxvb{0, 0, 0}{0}}}{\huxtpad{6pt + 1em}\raggedleft \hspace{6pt} {\fontsize{12pt}{14.4pt}\selectfont 4045.333 ***} \hspace{6pt}\huxbpad{6pt}} \tabularnewline[-0.5pt]


\hhline{}
\arrayrulecolor{black}

\multicolumn{1}{!{\huxvb{0, 0, 0}{0}}l!{\huxvb{0, 0, 0}{0}}}{\huxtpad{6pt + 1em}\raggedright \hspace{6pt} {\fontsize{12pt}{14.4pt}\selectfont } \hspace{6pt}\huxbpad{6pt}} &
\multicolumn{1}{r!{\huxvb{0, 0, 0}{0}}}{\huxtpad{6pt + 1em}\raggedleft \hspace{6pt} {\fontsize{12pt}{14.4pt}\selectfont (13.055)\hphantom{0}\hphantom{0}\hphantom{0}} \hspace{6pt}\huxbpad{6pt}} &
\multicolumn{1}{r!{\huxvb{0, 0, 0}{0}}}{\huxtpad{6pt + 1em}\raggedleft \hspace{6pt} {\fontsize{12pt}{14.4pt}\selectfont (740.556)\hphantom{0}\hphantom{0}\hphantom{0}} \hspace{6pt}\huxbpad{6pt}} &
\multicolumn{1}{r!{\huxvb{0, 0, 0}{0}}}{\huxtpad{6pt + 1em}\raggedleft \hspace{6pt} {\fontsize{12pt}{14.4pt}\selectfont (286.205)\hphantom{0}\hphantom{0}\hphantom{0}} \hspace{6pt}\huxbpad{6pt}} \tabularnewline[-0.5pt]


\hhline{}
\arrayrulecolor{black}

\multicolumn{1}{!{\huxvb{0, 0, 0}{0}}l!{\huxvb{0, 0, 0}{0}}}{\huxtpad{6pt + 1em}\raggedright \hspace{6pt} {\fontsize{12pt}{14.4pt}\selectfont carat} \hspace{6pt}\huxbpad{6pt}} &
\multicolumn{1}{r!{\huxvb{0, 0, 0}{0}}}{\huxtpad{6pt + 1em}\raggedleft \hspace{6pt} {\fontsize{12pt}{14.4pt}\selectfont 7756.426 ***} \hspace{6pt}\huxbpad{6pt}} &
\multicolumn{1}{r!{\huxvb{0, 0, 0}{0}}}{\huxtpad{6pt + 1em}\raggedleft \hspace{6pt} {\fontsize{12pt}{14.4pt}\selectfont \hphantom{0}\hphantom{0}\hphantom{0}\hphantom{0}\hphantom{0}\hphantom{0}\hphantom{0}\hphantom{0}} \hspace{6pt}\huxbpad{6pt}} &
\multicolumn{1}{r!{\huxvb{0, 0, 0}{0}}}{\huxtpad{6pt + 1em}\raggedleft \hspace{6pt} {\fontsize{12pt}{14.4pt}\selectfont 7765.141 ***} \hspace{6pt}\huxbpad{6pt}} \tabularnewline[-0.5pt]


\hhline{}
\arrayrulecolor{black}

\multicolumn{1}{!{\huxvb{0, 0, 0}{0}}l!{\huxvb{0, 0, 0}{0}}}{\huxtpad{6pt + 1em}\raggedright \hspace{6pt} {\fontsize{12pt}{14.4pt}\selectfont } \hspace{6pt}\huxbpad{6pt}} &
\multicolumn{1}{r!{\huxvb{0, 0, 0}{0}}}{\huxtpad{6pt + 1em}\raggedleft \hspace{6pt} {\fontsize{12pt}{14.4pt}\selectfont (14.067)\hphantom{0}\hphantom{0}\hphantom{0}} \hspace{6pt}\huxbpad{6pt}} &
\multicolumn{1}{r!{\huxvb{0, 0, 0}{0}}}{\huxtpad{6pt + 1em}\raggedleft \hspace{6pt} {\fontsize{12pt}{14.4pt}\selectfont \hphantom{0}\hphantom{0}\hphantom{0}\hphantom{0}\hphantom{0}\hphantom{0}\hphantom{0}\hphantom{0}} \hspace{6pt}\huxbpad{6pt}} &
\multicolumn{1}{r!{\huxvb{0, 0, 0}{0}}}{\huxtpad{6pt + 1em}\raggedleft \hspace{6pt} {\fontsize{12pt}{14.4pt}\selectfont (14.009)\hphantom{0}\hphantom{0}\hphantom{0}} \hspace{6pt}\huxbpad{6pt}} \tabularnewline[-0.5pt]


\hhline{}
\arrayrulecolor{black}

\multicolumn{1}{!{\huxvb{0, 0, 0}{0}}l!{\huxvb{0, 0, 0}{0}}}{\huxtpad{6pt + 1em}\raggedright \hspace{6pt} {\fontsize{12pt}{14.4pt}\selectfont depth} \hspace{6pt}\huxbpad{6pt}} &
\multicolumn{1}{r!{\huxvb{0, 0, 0}{0}}}{\huxtpad{6pt + 1em}\raggedleft \hspace{6pt} {\fontsize{12pt}{14.4pt}\selectfont \hphantom{0}\hphantom{0}\hphantom{0}\hphantom{0}\hphantom{0}\hphantom{0}\hphantom{0}\hphantom{0}} \hspace{6pt}\huxbpad{6pt}} &
\multicolumn{1}{r!{\huxvb{0, 0, 0}{0}}}{\huxtpad{6pt + 1em}\raggedleft \hspace{6pt} {\fontsize{12pt}{14.4pt}\selectfont -29.650 *\hphantom{0}\hphantom{0}} \hspace{6pt}\huxbpad{6pt}} &
\multicolumn{1}{r!{\huxvb{0, 0, 0}{0}}}{\huxtpad{6pt + 1em}\raggedleft \hspace{6pt} {\fontsize{12pt}{14.4pt}\selectfont -102.165 ***} \hspace{6pt}\huxbpad{6pt}} \tabularnewline[-0.5pt]


\hhline{}
\arrayrulecolor{black}

\multicolumn{1}{!{\huxvb{0, 0, 0}{0}}l!{\huxvb{0, 0, 0}{0}}}{\huxtpad{6pt + 1em}\raggedright \hspace{6pt} {\fontsize{12pt}{14.4pt}\selectfont } \hspace{6pt}\huxbpad{6pt}} &
\multicolumn{1}{r!{\huxvb{0, 0, 0}{0}}}{\huxtpad{6pt + 1em}\raggedleft \hspace{6pt} {\fontsize{12pt}{14.4pt}\selectfont \hphantom{0}\hphantom{0}\hphantom{0}\hphantom{0}\hphantom{0}\hphantom{0}\hphantom{0}\hphantom{0}} \hspace{6pt}\huxbpad{6pt}} &
\multicolumn{1}{r!{\huxvb{0, 0, 0}{0}}}{\huxtpad{6pt + 1em}\raggedleft \hspace{6pt} {\fontsize{12pt}{14.4pt}\selectfont (11.990)\hphantom{0}\hphantom{0}\hphantom{0}} \hspace{6pt}\huxbpad{6pt}} &
\multicolumn{1}{r!{\huxvb{0, 0, 0}{0}}}{\huxtpad{6pt + 1em}\raggedleft \hspace{6pt} {\fontsize{12pt}{14.4pt}\selectfont (4.635)\hphantom{0}\hphantom{0}\hphantom{0}} \hspace{6pt}\huxbpad{6pt}} \tabularnewline[-0.5pt]


\hhline{>{\huxb{255, 255, 255}{0.4}}->{\huxb{0, 0, 0}{0.4}}->{\huxb{0, 0, 0}{0.4}}->{\huxb{0, 0, 0}{0.4}}-}
\arrayrulecolor{black}

\multicolumn{1}{!{\huxvb{0, 0, 0}{0}}l!{\huxvb{0, 0, 0}{0}}}{\huxtpad{6pt + 1em}\raggedright \hspace{6pt} {\fontsize{12pt}{14.4pt}\selectfont N} \hspace{6pt}\huxbpad{6pt}} &
\multicolumn{1}{r!{\huxvb{0, 0, 0}{0}}}{\huxtpad{6pt + 1em}\raggedleft \hspace{6pt} {\fontsize{12pt}{14.4pt}\selectfont 53940\hphantom{0}\hphantom{0}\hphantom{0}\hphantom{0}\hphantom{0}\hphantom{0}\hphantom{0}\hphantom{0}} \hspace{6pt}\huxbpad{6pt}} &
\multicolumn{1}{r!{\huxvb{0, 0, 0}{0}}}{\huxtpad{6pt + 1em}\raggedleft \hspace{6pt} {\fontsize{12pt}{14.4pt}\selectfont 53940\hphantom{0}\hphantom{0}\hphantom{0}\hphantom{0}\hphantom{0}\hphantom{0}\hphantom{0}\hphantom{0}} \hspace{6pt}\huxbpad{6pt}} &
\multicolumn{1}{r!{\huxvb{0, 0, 0}{0}}}{\huxtpad{6pt + 1em}\raggedleft \hspace{6pt} {\fontsize{12pt}{14.4pt}\selectfont 53940\hphantom{0}\hphantom{0}\hphantom{0}\hphantom{0}\hphantom{0}\hphantom{0}\hphantom{0}\hphantom{0}} \hspace{6pt}\huxbpad{6pt}} \tabularnewline[-0.5pt]


\hhline{}
\arrayrulecolor{black}

\multicolumn{1}{!{\huxvb{0, 0, 0}{0}}l!{\huxvb{0, 0, 0}{0}}}{\huxtpad{6pt + 1em}\raggedright \hspace{6pt} {\fontsize{12pt}{14.4pt}\selectfont R2} \hspace{6pt}\huxbpad{6pt}} &
\multicolumn{1}{r!{\huxvb{0, 0, 0}{0}}}{\huxtpad{6pt + 1em}\raggedleft \hspace{6pt} {\fontsize{12pt}{14.4pt}\selectfont 0.849\hphantom{0}\hphantom{0}\hphantom{0}\hphantom{0}} \hspace{6pt}\huxbpad{6pt}} &
\multicolumn{1}{r!{\huxvb{0, 0, 0}{0}}}{\huxtpad{6pt + 1em}\raggedleft \hspace{6pt} {\fontsize{12pt}{14.4pt}\selectfont 0.000\hphantom{0}\hphantom{0}\hphantom{0}\hphantom{0}} \hspace{6pt}\huxbpad{6pt}} &
\multicolumn{1}{r!{\huxvb{0, 0, 0}{0}}}{\huxtpad{6pt + 1em}\raggedleft \hspace{6pt} {\fontsize{12pt}{14.4pt}\selectfont 0.851\hphantom{0}\hphantom{0}\hphantom{0}\hphantom{0}} \hspace{6pt}\huxbpad{6pt}} \tabularnewline[-0.5pt]


\hhline{>{\huxb{0, 0, 0}{0.8}}->{\huxb{0, 0, 0}{0.8}}->{\huxb{0, 0, 0}{0.8}}->{\huxb{0, 0, 0}{0.8}}-}
\arrayrulecolor{black}

\multicolumn{4}{!{\huxvb{0, 0, 0}{0}}l!{\huxvb{0, 0, 0}{0}}}{\huxtpad{6pt + 1em}\raggedright \hspace{6pt} {\fontsize{12pt}{14.4pt}\selectfont  *** p $<$ 0.001;  ** p $<$ 0.01;  * p $<$ 0.05.} \hspace{6pt}\huxbpad{6pt}} \tabularnewline[-0.5pt]


\hhline{}
\arrayrulecolor{black}
\end{tabular}
\end{threeparttable}\par\end{centerbox}

\end{table}
 

FYI - R also recently introduced the gt package, which is worthwhile
exploring too.

\hypertarget{lists}{%
\section{Lists}\label{lists}}

To add lists, simply using the following notation

\begin{itemize}
\item
  This is really simple

  \begin{itemize}
  \tightlist
  \item
    Just note the spaces here - writing in R you have to sometimes be
    pedantic about spaces\ldots{}
  \end{itemize}
\item
  Note that Rmarkdown notation removes the pain of defining
  \LaTeX environments!
\end{itemize}

\hypertarget{conclusion}{%
\section{Conclusion}\label{conclusion}}

I hope you find this template useful. Remember, stackoverflow is your
friend - use it to find answers to questions. Feel free to write me a
mail if you have any questions regarding the use of this package. To
cite this package, simply type citation(``Texevier'') in Rstudio to get
the citation for \protect\hyperlink{ref-Texevier}{Katzke}
(\protect\hyperlink{ref-Texevier}{2017}) (Note that uncited references
in your bibtex file will not be included in References).

\newpage

\hypertarget{references}{%
\section*{References}\label{references}}
\addcontentsline{toc}{section}{References}

\hypertarget{refs}{}
\begin{CSLReferences}{1}{0}
\leavevmode\vadjust pre{\hypertarget{ref-fama1997}{}}%
Fama, E.F. \& French, K.R. 1997. Industry costs of equity. \emph{Journal
of financial economics}. 43(2):153--193.

\leavevmode\vadjust pre{\hypertarget{ref-grinold2000}{}}%
Grinold, R.C. \& Kahn, R.N. 2000. Active portfolio management.

\leavevmode\vadjust pre{\hypertarget{ref-Texevier}{}}%
Katzke, N.F. 2017. \emph{{Texevier}: {P}ackage to create elsevier
templates for rmarkdown}. Stellenbosch, South Africa: Bureau for
Economic Research.

\end{CSLReferences}

\hypertarget{appendix}{%
\section*{Appendix}\label{appendix}}
\addcontentsline{toc}{section}{Appendix}

\hypertarget{appendix-a}{%
\subsection*{Appendix A}\label{appendix-a}}
\addcontentsline{toc}{subsection}{Appendix A}

Some appendix information here

\hypertarget{appendix-b}{%
\subsection*{Appendix B}\label{appendix-b}}
\addcontentsline{toc}{subsection}{Appendix B}

\bibliography{Tex/ref}





\end{document}
